\documentclass[ignorenonframetext,]{beamer}
\setbeamertemplate{caption}[numbered]
\setbeamertemplate{caption label separator}{: }
\setbeamercolor{caption name}{fg=normal text.fg}
\beamertemplatenavigationsymbolsempty
\usepackage{lmodern}
\usepackage{amssymb,amsmath}
\usepackage{ifxetex,ifluatex}
\usepackage{fixltx2e} % provides \textsubscript
\ifnum 0\ifxetex 1\fi\ifluatex 1\fi=0 % if pdftex
  \usepackage[T1]{fontenc}
  \usepackage[utf8]{inputenc}
\else % if luatex or xelatex
  \ifxetex
    \usepackage{mathspec}
  \else
    \usepackage{fontspec}
  \fi
  \defaultfontfeatures{Ligatures=TeX,Scale=MatchLowercase}
\fi
\usefonttheme{structurebold}
% use upquote if available, for straight quotes in verbatim environments
\IfFileExists{upquote.sty}{\usepackage{upquote}}{}
% use microtype if available
\IfFileExists{microtype.sty}{%
\usepackage{microtype}
\UseMicrotypeSet[protrusion]{basicmath} % disable protrusion for tt fonts
}{}
\newif\ifbibliography
\usepackage{color}
\usepackage{fancyvrb}
\newcommand{\VerbBar}{|}
\newcommand{\VERB}{\Verb[commandchars=\\\{\}]}
\DefineVerbatimEnvironment{Highlighting}{Verbatim}{commandchars=\\\{\}}
% Add ',fontsize=\small' for more characters per line
\usepackage{framed}
\definecolor{shadecolor}{RGB}{248,248,248}
\newenvironment{Shaded}{\begin{snugshade}}{\end{snugshade}}
\newcommand{\KeywordTok}[1]{\textcolor[rgb]{0.13,0.29,0.53}{\textbf{{#1}}}}
\newcommand{\DataTypeTok}[1]{\textcolor[rgb]{0.13,0.29,0.53}{{#1}}}
\newcommand{\DecValTok}[1]{\textcolor[rgb]{0.00,0.00,0.81}{{#1}}}
\newcommand{\BaseNTok}[1]{\textcolor[rgb]{0.00,0.00,0.81}{{#1}}}
\newcommand{\FloatTok}[1]{\textcolor[rgb]{0.00,0.00,0.81}{{#1}}}
\newcommand{\ConstantTok}[1]{\textcolor[rgb]{0.00,0.00,0.00}{{#1}}}
\newcommand{\CharTok}[1]{\textcolor[rgb]{0.31,0.60,0.02}{{#1}}}
\newcommand{\SpecialCharTok}[1]{\textcolor[rgb]{0.00,0.00,0.00}{{#1}}}
\newcommand{\StringTok}[1]{\textcolor[rgb]{0.31,0.60,0.02}{{#1}}}
\newcommand{\VerbatimStringTok}[1]{\textcolor[rgb]{0.31,0.60,0.02}{{#1}}}
\newcommand{\SpecialStringTok}[1]{\textcolor[rgb]{0.31,0.60,0.02}{{#1}}}
\newcommand{\ImportTok}[1]{{#1}}
\newcommand{\CommentTok}[1]{\textcolor[rgb]{0.56,0.35,0.01}{\textit{{#1}}}}
\newcommand{\DocumentationTok}[1]{\textcolor[rgb]{0.56,0.35,0.01}{\textbf{\textit{{#1}}}}}
\newcommand{\AnnotationTok}[1]{\textcolor[rgb]{0.56,0.35,0.01}{\textbf{\textit{{#1}}}}}
\newcommand{\CommentVarTok}[1]{\textcolor[rgb]{0.56,0.35,0.01}{\textbf{\textit{{#1}}}}}
\newcommand{\OtherTok}[1]{\textcolor[rgb]{0.56,0.35,0.01}{{#1}}}
\newcommand{\FunctionTok}[1]{\textcolor[rgb]{0.00,0.00,0.00}{{#1}}}
\newcommand{\VariableTok}[1]{\textcolor[rgb]{0.00,0.00,0.00}{{#1}}}
\newcommand{\ControlFlowTok}[1]{\textcolor[rgb]{0.13,0.29,0.53}{\textbf{{#1}}}}
\newcommand{\OperatorTok}[1]{\textcolor[rgb]{0.81,0.36,0.00}{\textbf{{#1}}}}
\newcommand{\BuiltInTok}[1]{{#1}}
\newcommand{\ExtensionTok}[1]{{#1}}
\newcommand{\PreprocessorTok}[1]{\textcolor[rgb]{0.56,0.35,0.01}{\textit{{#1}}}}
\newcommand{\AttributeTok}[1]{\textcolor[rgb]{0.77,0.63,0.00}{{#1}}}
\newcommand{\RegionMarkerTok}[1]{{#1}}
\newcommand{\InformationTok}[1]{\textcolor[rgb]{0.56,0.35,0.01}{\textbf{\textit{{#1}}}}}
\newcommand{\WarningTok}[1]{\textcolor[rgb]{0.56,0.35,0.01}{\textbf{\textit{{#1}}}}}
\newcommand{\AlertTok}[1]{\textcolor[rgb]{0.94,0.16,0.16}{{#1}}}
\newcommand{\ErrorTok}[1]{\textcolor[rgb]{0.64,0.00,0.00}{\textbf{{#1}}}}
\newcommand{\NormalTok}[1]{{#1}}

% Prevent slide breaks in the middle of a paragraph:
\widowpenalties 1 10000
\raggedbottom

\AtBeginPart{
  \let\insertpartnumber\relax
  \let\partname\relax
  \frame{\partpage}
}
\AtBeginSection{
  \ifbibliography
  \else
    \let\insertsectionnumber\relax
    \let\sectionname\relax
    \frame{\sectionpage}
  \fi
}
\AtBeginSubsection{
  \let\insertsubsectionnumber\relax
  \let\subsectionname\relax
  \frame{\subsectionpage}
}

\setlength{\emergencystretch}{3em}  % prevent overfull lines
\providecommand{\tightlist}{%
  \setlength{\itemsep}{0pt}\setlength{\parskip}{0pt}}
\setcounter{secnumdepth}{0}
\definecolor{links}{HTML}{800080}
\hypersetup{colorlinks,linkcolor=,urlcolor=links}

\title{Web Data Collection with R}
\subtitle{Character Encoding}
\author{Peter Meißner / 2016-02-29 -- 2016-03-04 / ECPR WSMT}
\date{}

\begin{document}
\frame{\titlepage}

\begin{frame}
\tableofcontents[hideallsubsections]
\end{frame}

\section{Character Encodings}\label{character-encodings}

\begin{frame}{Character Encodings}

Character Encodings are \ldots{}

\begin{itemize}
\tightlist
\item
  are like family \ldots{}
\item
  \ldots{} some of them you do not like but cannot avoid \ldots{}
\item
  \ldots{} something we will struggle with but have cope anyways
\end{itemize}

The best thing is \ldots{}

\begin{itemize}
\tightlist
\item
  R has them all
\end{itemize}

The worst thing is \ldots{}

\begin{itemize}
\tightlist
\item
  R has them all
\end{itemize}

\end{frame}

\begin{frame}{Character Encodings}

\begin{itemize}
\tightlist
\item
  computers store everything as 0s and 1s (bits)
\item
  in cs there are differing layers of abstraction
\item
  one bit of information is called bit
\item
  bits are quite uninformative as they only ave two states
\item
  so they are are grouped into bytes (8 bits)
\item
  one byte can have 256 different values (2\^{}8)
\item
  so it can store numbers 0 to 255 or 1 to 256 or \ldots{} -127 to 128
\item
  or it can map to characters e.g.~ASCII
  (abcABC.:-\_,;\#'+*\textasciitilde{}\textbar{}\textless{}\textgreater{}!``§\$\%\&/()=?\}{]}{[}\{\}\^{}°,
  \ldots{}'')
\item
  ASCII is a character set - the set of characters you want to be able
  to store - even 7 Bits would suffice to store it
\end{itemize}

\end{frame}

\begin{frame}{Character Encodings}

\begin{itemize}
\tightlist
\item
  for larger character sets than ASCII (ä ö ü é è \ldots{} ) on needs to
  get clever since one byte does not suffice to map all characters to 0s
  and 1s
\item
  unfortunate people got clever in differing ways

  \begin{enumerate}
  \def\labelenumi{\arabic{enumi})}
  \tightlist
  \item
    using more than one byte to map more characters (`wide' characters,
    UTF-16, USC-2, Windows OSs)
  \item
    using one or more bytes and using the first byte to encode how
    manies are used (`multi-byte characters', UTF-8, Unix based OSs)
  \end{enumerate}
\item
  otherwise we would not have to talk about character sets and character
  encodings
\end{itemize}

\end{frame}

\begin{frame}[fragile]{Character Encodings}

\begin{Shaded}
\begin{Highlighting}[]
\KeywordTok{rawToBits}\NormalTok{(}\KeywordTok{as.raw}\NormalTok{(}\DecValTok{62}\NormalTok{:}\DecValTok{66}\NormalTok{)) }\CommentTok{# as bits}
\end{Highlighting}
\end{Shaded}

\begin{verbatim}
##  [1] 00 01 01 01 01 01 00 00 01 01 01 01 01 01 00 00 00 00 00 00 00 00 01
## [24] 00 01 00 00 00 00 00 01 00 00 01 00 00 00 00 01 00
\end{verbatim}

\begin{Shaded}
\begin{Highlighting}[]
\KeywordTok{as.raw}\NormalTok{(}\DecValTok{62}\NormalTok{:}\DecValTok{66}\NormalTok{) }\CommentTok{# bytes as hexa-decimal}
\end{Highlighting}
\end{Shaded}

\begin{verbatim}
## [1] 3e 3f 40 41 42
\end{verbatim}

\begin{Shaded}
\begin{Highlighting}[]
\KeywordTok{as.numeric}\NormalTok{(}\KeywordTok{as.raw}\NormalTok{(}\DecValTok{62}\NormalTok{:}\DecValTok{66}\NormalTok{)) }\CommentTok{# as numbers}
\end{Highlighting}
\end{Shaded}

\begin{verbatim}
## [1] 62 63 64 65 66
\end{verbatim}

\begin{Shaded}
\begin{Highlighting}[]
\KeywordTok{rawToChar}\NormalTok{(}\KeywordTok{as.raw}\NormalTok{(}\DecValTok{62}\NormalTok{:}\DecValTok{66}\NormalTok{)) }\CommentTok{# bytes as chararcters}
\end{Highlighting}
\end{Shaded}

\begin{verbatim}
## [1] ">?@AB"
\end{verbatim}

\end{frame}

\begin{frame}[fragile]{A character set problem}

\begin{Shaded}
\begin{Highlighting}[]
\NormalTok{text           <-}\StringTok{ }\KeywordTok{rawToChar}\NormalTok{(}\KeywordTok{as.raw}\NormalTok{(}\DecValTok{228}\NormalTok{))}
\KeywordTok{Encoding}\NormalTok{(text) <-}\StringTok{ "UTF-8"} 
\NormalTok{text}
\end{Highlighting}
\end{Shaded}

\begin{verbatim}
## [1] "\xe4"
\end{verbatim}

\begin{Shaded}
\begin{Highlighting}[]
\KeywordTok{Encoding}\NormalTok{(text) <-}\StringTok{ "latin1"} 
\NormalTok{text}
\end{Highlighting}
\end{Shaded}

\begin{verbatim}
## [1] "ä"
\end{verbatim}

Results differ because for latin1 character 228 is know but not for
UTF-8

\end{frame}

\begin{frame}[fragile]{An encoding problem}

Of cause UTF-8 knows how to encode ``ä'' \ldots{}

\begin{Shaded}
\begin{Highlighting}[]
\NormalTok{text <-}\StringTok{ "ä"}
\KeywordTok{charToRaw}\NormalTok{(text)}
\end{Highlighting}
\end{Shaded}

\begin{verbatim}
## [1] c3 a4
\end{verbatim}

\begin{Shaded}
\begin{Highlighting}[]
\KeywordTok{Encoding}\NormalTok{(text) <-}\StringTok{ "latin1"}
\NormalTok{text}
\end{Highlighting}
\end{Shaded}

\begin{verbatim}
## [1] "ä"
\end{verbatim}

\ldots{} but here the results differ because ``UTF-8'' has another
system translating characters to bytes. In latin1 the two bytes are
interpreted as two characters.

\end{frame}

\begin{frame}[fragile]{Which default encoding does your R use}

\begin{Shaded}
\begin{Highlighting}[]
\KeywordTok{Sys.getlocale}\NormalTok{()}
\end{Highlighting}
\end{Shaded}

\begin{verbatim}
## [1] "LC_CTYPE=de_DE.UTF-8;LC_NUMERIC=C;LC_TIME=de_DE.UTF-8;LC_COLLATE=de_DE.UTF-8;LC_MONETARY=de_DE.UTF-8;LC_MESSAGES=de_DE.UTF-8;LC_PAPER=de_DE.UTF-8;LC_NAME=C;LC_ADDRESS=C;LC_TELEPHONE=C;LC_MEASUREMENT=de_DE.UTF-8;LC_IDENTIFICATION=C"
\end{verbatim}

\begin{Shaded}
\begin{Highlighting}[]
\CommentTok{# if yor locale is something other than UTF-8, }
\CommentTok{# switch 'latin1' and 'UTF-8' and you shall be good to go}
\end{Highlighting}
\end{Shaded}

\end{frame}

\begin{frame}[fragile]{Changing interpretation of bytes}

\begin{Shaded}
\begin{Highlighting}[]
\NormalTok{text <-}\StringTok{ "Små grodorna, små grodorna är lustiga att se."}
\KeywordTok{Encoding}\NormalTok{(text) <-}\StringTok{ "UTF-8"}
\NormalTok{text}
\end{Highlighting}
\end{Shaded}

\begin{verbatim}
## [1] "Små grodorna, små grodorna är lustiga att se."
\end{verbatim}

\end{frame}

\begin{frame}[fragile]{Changing interpretation of bytes}

\begin{Shaded}
\begin{Highlighting}[]
\NormalTok{text <-}\StringTok{ "Små grodorna, små grodorna är lustiga att se."}
\KeywordTok{Encoding}\NormalTok{(text) <-}\StringTok{ "latin1"}
\NormalTok{text}
\end{Highlighting}
\end{Shaded}

\begin{verbatim}
## [1] "Små grodorna, små grodorna är lustiga att se."
\end{verbatim}

\end{frame}

\begin{frame}[fragile]{Changing bytes and interpretation}

\begin{Shaded}
\begin{Highlighting}[]
\NormalTok{text <-}\StringTok{ "Små grodorna, små grodorna är lustiga att se."}
\NormalTok{text <-}\StringTok{ }\KeywordTok{iconv}\NormalTok{(text, }\StringTok{"UTF-8"}\NormalTok{, }\StringTok{"latin1"}\NormalTok{)}
\KeywordTok{Encoding}\NormalTok{(text)}
\end{Highlighting}
\end{Shaded}

\begin{verbatim}
## [1] "latin1"
\end{verbatim}

\begin{Shaded}
\begin{Highlighting}[]
\NormalTok{text}
\end{Highlighting}
\end{Shaded}

\begin{verbatim}
## [1] "Små grodorna, små grodorna är lustiga att se."
\end{verbatim}

\end{frame}

\begin{frame}[fragile]{Noe that all sources might have another encoding
than your R default locale!}

\begin{Shaded}
\begin{Highlighting}[]
\NormalTok{text <-}\StringTok{ "Små grodorna, små grodorna är lustiga att se."}
\NormalTok{text <-}\StringTok{ }\KeywordTok{iconv}\NormalTok{(text, }\StringTok{"UTF-8"}\NormalTok{, }\StringTok{"latin1"}\NormalTok{)}
\KeywordTok{writeLines}\NormalTok{(text, }\StringTok{"text_latin1.txt"}\NormalTok{, }\DataTypeTok{useBytes =} \OtherTok{TRUE}\NormalTok{)}
\NormalTok{text <-}\StringTok{ }\KeywordTok{readLines}\NormalTok{(}\StringTok{"text_latin1.txt"}\NormalTok{)}
\KeywordTok{Encoding}\NormalTok{(text)}
\end{Highlighting}
\end{Shaded}

\begin{verbatim}
## [1] "unknown"
\end{verbatim}

\begin{Shaded}
\begin{Highlighting}[]
\NormalTok{text}
\end{Highlighting}
\end{Shaded}

\begin{verbatim}
## [1] "Sm\xe5 grodorna, sm\xe5 grodorna \xe4r lustiga att se."
\end{verbatim}

\begin{Shaded}
\begin{Highlighting}[]
\KeywordTok{Encoding}\NormalTok{(text) <-}\StringTok{ "latin1"}
\NormalTok{text}
\end{Highlighting}
\end{Shaded}

\begin{verbatim}
## [1] "Små grodorna, små grodorna är lustiga att se."
\end{verbatim}

\end{frame}

\end{document}
